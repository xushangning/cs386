\documentclass[journal]{IEEEtran}


% correct bad hyphenation here
\hyphenation{op-tical net-works semi-conduc-tor}
\usepackage{amsmath}
\usepackage{comment}
\usepackage{booktabs}
\usepackage{subfigure}
\usepackage{xcolor}
\usepackage{listings}
\lstset{
  basicstyle=\fontsize{9}{10}\selectfont\ttfamily,
  numbers=left,
  numberstyle= \tiny,
  keywordstyle= \color{ blue!70},
  commentstyle= \color{red!50!green!50!blue!50},
  frame=single,
  rulesepcolor= \color{ red!20!green!20!blue!20} ,
  escapeinside=``,
  xleftmargin=1.5em,xrightmargin=0em, aboveskip=1em,
  framexleftmargin=2em,
  showstringspaces=false,
  showtabs=false,
  breaklines=true
}
\lstdefinelanguage{Solidity}
{
  morekeywords={contract, mapping, address, uint, private, function, public, if, payable},
  morecomment=[l]{//},
  morestring=[b]"
}


\usepackage{multicol}
\usepackage{lipsum}
\usepackage{mathtools}
\usepackage{cuted}

\usepackage{amsmath}
\usepackage{extpfeil}
\usepackage{mathpartir}
\usepackage[mathscr]{eucal}

\usepackage{hyperref}
\usepackage{cleveref}

\crefformat{section}{\S#2#1#3} % see manual of cleveref, section 8.2.1
\crefformat{subsection}{\S#2#1#3}
\crefformat{subsubsection}{\S#2#1#3}

\begin{document}
%
% paper title
% Titles are generally capitalized except for words such as a, an, and, as,
% at, but, by, for, in, nor, of, on, or, the, to and up, which are usually
% not capitalized unless they are the first or last word of the title.
% Linebreaks \\ can be used within to get better formatting as desired.
% Do not put math or special symbols in the title.
\title{Bare Demo of IEEEtran.cls for Journals}
%
%
% author names and IEEE memberships
% note positions of commas and nonbreaking spaces ( ~ ) LaTeX will not break
% a structure at a ~ so this keeps an author's name from being broken across
% two lines.
% use \thanks{} to gain access to the first footnote area
% a separate \thanks must be used for each paragraph as LaTeX2e's \thanks
% was not built to handle multiple paragraphs
%

\author{Michael~Shell,~\IEEEmembership{Member,~IEEE,}
	John~Doe,~\IEEEmembership{Fellow,~OSA,}
	and~Jane~Doe,~\IEEEmembership{Life~Fellow,~IEEE}% <-this % stops a space
	\thanks{M. Shell is with the Department
		of Electrical and Computer Engineering, Georgia Institute of Technology, Atlanta,
		GA, 30332 USA e-mail: (see http://www.michaelshell.org/contact.html).}% <-this % stops a space
	\thanks{J. Doe and J. Doe are with Anonymous University.}% <-this % stops a space
	\thanks{Manuscript received April 19, 2005; revised September 17, 2014.}}

% The paper headers
\markboth{Journal of \LaTeX\ Class Files,~Vol.~13, No.~9, September~2014}%
{Shell \MakeLowercase{\textit{et al.}}: Bare Demo of IEEEtran.cls for Journals}
% The only time the second header will appear is for the odd numbered pages
% after the title page when using the twoside option.
%
% *** Note that you probably will NOT want to include the author's ***
% *** name in the headers of peer review papers.                   ***
% You can use \ifCLASSOPTIONpeerreview for conditional compilation here if
% you desire.


% make the title area
\maketitle

% As a general rule, do not put math, special symbols or citations
% in the abstract or keywords.
\begin{abstract}
	The abstract goes here.
\end{abstract}

% Note that keywords are not normally used for peerreview papers.
\begin{IEEEkeywords}
	IEEEtran, journal, \LaTeX, paper, template.
\end{IEEEkeywords}


\IEEEpeerreviewmaketitle



\section{Introduction}
\IEEEPARstart{T}{his}

% wzy
\section{Data Augmentation}
The original dataset contains 40 micro categories of oracle bone characters, each with 85 samples.
Therefore, we have $40 \times 85 = 3400$ samples in total.
This is barely enough for training a convolutional neural network without overfitting, and we need to apply data augmentation to obtain larger dataset adequate for training.

We first make the following observations.

\paragraph{Observation 1} For each pair of characters in Figure~\ref{fig:sym-across}, we may consider them horizontally symmetric.
Thus we can double the number of samples by flipping characters in one category horizontally to obtain those in another category.

\begin{figure}[h]
	\centering
	\begin{minipage}{0.2\linewidth}
		\includegraphics[width=0.8\linewidth]{fig/observation-1-1.png}
		\includegraphics[width=0.8\linewidth]{fig/observation-1-2.png}
	\end{minipage}
	% \hfill
	\begin{minipage}{0.2\linewidth}
		\includegraphics[width=0.8\linewidth]{fig/observation-1-3.png}
		\includegraphics[width=0.8\linewidth]{fig/observation-1-4.png}
	\end{minipage}
	% \hfill
	\begin{minipage}{0.2\linewidth}
		\includegraphics[width=0.8\linewidth]{fig/observation-1-5.png}
		\includegraphics[width=0.8\linewidth]{fig/observation-1-6.png}
	\end{minipage}
	\caption{Symmetric Samples Across Micro-Categories}
	\label{fig:sym-across}
\end{figure}

\paragraph{Observation 2} For characters in Figure~\ref{fig:sym-in}, they are horizontally symmetric by themselves.
We can flip them horizontally to double the number of samples in these categories.

\begin{figure}[h]
	\centering
	\includegraphics[width=0.8\linewidth]{fig/observation-2.png}
	\caption{Symmetric Samples in Micro-Categories}
	\label{fig:sym-in}
\end{figure}

To balance the samples in different categories after applying above two methods of data augmentation, we need to also double the number of samples in the rest of categories.
We choose to apply affine transforms to the remaining images.
More specifically, we randomly scale images up/down by 0 to 10 percent, randomly translate the image horizontally/vertically by 0 to 10 pixels, randomly rotate and shear images by -10 to 10 degrees.
The augmented images is with the same categories as the original ones.

Then, we need to introduce more noises into the dataset to allow the model trained on it generalize well.
We apply gaussian blur to original images, randomly strengthen/weaken the contrast in each images, randomly add some gaussian noise to each pixel or one channel of images, and randomly make some images brighter or darker by multiplying channels of images.

Figure~\ref{fig:aug-example} shows an example of the affine augmentation and the noise augmentation.
\begin{figure}[h]
	\centering
	\begin{minipage}{0.2\linewidth}
		\subfigure[Original]{
			\includegraphics[width=\linewidth]{fig/person_0001.jpg}
		}
	\end{minipage}
	\begin{minipage}{0.2\linewidth}
		\subfigure[Affine]{
			\includegraphics[width=\linewidth]{fig/person_0001_affine.jpg}
		}
	\end{minipage}
	\begin{minipage}{0.2\linewidth}
		\subfigure[Noise]{
			\includegraphics[width=\linewidth]{fig/person_0001_noise.jpg}
		}
	\end{minipage}
	\caption{An Example of Affine Augmentation and Noise Augmentation}
	\label{fig:aug-example}
\end{figure}

When we actually apply the above augmentation methods to images, we first split the dataset into training set and validation set and only apply augment the samples in the training set to avoid data leakage.
We select 10 images from each micro categories to form the validation set.
Eventually, we end up tripling the size of the dataset with 9120 training samples and 400 validation samples.

\section{Character Recognition}
% wzy
\subsection{Convolutional Neural Network}
% ljy
\subsection{Template Matching}
% ljy
\subsection{Ensemble of Two Methods}

\section{Experiments}
\subsection{Test Set}
\subsection{Results}

\section{Conclusion}
The conclusion goes here.

\bibliographystyle{IEEEtran}
\bibliography{Ref}


% that's all folks
\end{document}


